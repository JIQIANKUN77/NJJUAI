\documentclass[a4paper,UTF8]{article}
\usepackage{ctex}
\usepackage[margin=1.25in]{geometry}
\usepackage{color}
\usepackage{graphicx}
\usepackage{amssymb}
\usepackage{amsmath}
\usepackage{amsthm}
\usepackage{soul, color, xcolor}
\usepackage{bm}
\usepackage{tcolorbox}
\usepackage{hyperref}
\numberwithin{equation}{section}
%\usepackage[thmmarks, amsmath, thref]{ntheorem}
\theoremstyle{definition}
\newtheorem*{solution}{Solution}
\newtheorem*{prove}{Proof}
\usepackage{multirow}
\usepackage{diagbox}
\usepackage{float}

\def \X {\mathbf{X}}
\def \W {\mathbf{W}}
\def \A {\mathbf{A}}
\def \K {\mathbf{K}}
\def \B {\mathbf{B}}
\def \C {\mathbf{C}}
\def \D {\mathbf{D}}
\def \Q {\mathbf{Q}}
\def \S {\mathbf{S}}
\def \P {\mathbf{P}}
\def \Diag {\textbf{$\Lambda$}}
\def \w {\hat{\boldsymbol{w}}}
\def \y {\boldsymbol{y}}
\def \x {\boldsymbol{x}}
\def \z {\mathbf{z}}
\def \b {\mathbf{b}}
\def \by {\Bar{y}}
\def \H {\mathbf{H}}
\def \I {\mathbf{I}}
\setlength{\parindent}{0pt}
%--

%--
\begin{document}
\title{机器学习导论\ 习题三}
\author{221300066, 季千焜, \href{mailto:邮箱}{qkjiai@smail.nju.edu.cn}}
\maketitle
\section*{作业提交注意事项}
\begin{tcolorbox}
	\begin{enumerate}
    \item[1.] 作业所需的LaTeX及Python环境配置要求请参考: \href{https://www.lamda.nju.edu.cn/ML2024Spring/supplemantary/environment.pdf}{[Link]};
		\item[2.] 请在LaTeX模板中第一页填写个人的学号、姓名、邮箱;
		\item[3.] 本次作业需提交的文件与对应的命名方式为:
            \begin{enumerate}
                \item [(a)] 作答后的LaTeX代码 --- \texttt{HW3.tex};
                \item [(b)] 由(a)编译得到的PDF文件 --- \texttt{HW3.pdf};
                \item [(c)] 第三题模型代码 --- \texttt{p3\_models.py};
                \item [(d)] 第四题模型代码 --- \texttt{p4\_models.py};
                \item [(e)] 第四题训练代码 --- \texttt{p4\_trainer.py}.
            \end{enumerate}
            请将以上文件{\color{red}\textbf{打包为~学号\hspace{0em}\_\hspace{0em}姓名.zip}} (例如 221300001\hspace{0em}\_\hspace{0em}张三.zip) 后提交;
		\item[3.] 若多次提交作业, 则在命名~.zip 文件时加上版本号, 例如 221300001\_\hspace{0em}张三\hspace{0em}\_v1.zip” (批改时以版本号最高的文件为准);
		\item[4.] 本次作业提交截止时间为 {\color{red}\textbf{ 5 月 17 日23:59:59}}. 未按照要求提交作业, 提交作业格式不正确, {\color{red}\textbf{作业命名不规范}}, 将会被扣除部分作业分数; 除特殊原因 (如因病缓交, 需出示医院假条) 逾期未交作业, 本次作业记 0 分; {\color{red}\textbf{如发现抄袭, 抄袭和被抄袭双方成绩全部取消}};
        \item[5.] 学习过程中, 允许参考 ChatGPT 等生成式语言模型的生成结果, 但必须在可信的信息源处核实信息的真实性; {\color{red}\textbf{不允许直接使用模型的生成结果作为作业的回答内容}}, 否则将视为作业非本人完成并取消成绩;
		\item[6.] 本次作业提交地址为 \href{https://box.nju.edu.cn/u/d/cf36400095e94f769e1d/}{[Link]}, 请大家预留时间提前上交, 以防在临近截止日期时, 因网络等原因无法按时提交作业.
	\end{enumerate}
\end{tcolorbox}
\newpage


\section{[25pts] Principal Component Analysis}
主成分分析是一种经典且常用的数据降维方法. 请仔细阅读学习《机器学习》第十章 10.3 节, 并根据图 10.5 中的算法内容, 完成对如下 6 组样本数据的主成分分析. 
        \[
        \X = \begin{bmatrix}
            2&3&3&4&5&7\\
            2&4&5&5&6&8
        \end{bmatrix}
        \]
        
\begin{enumerate}
    \item[(1)] \textbf{[6pts]} 试求样本数据各维的均值、标准差.
    \item[(2)] \textbf{[7pts]} 试求\textcolor{blue}{标准化}后的样本矩阵 $\X_{\text{std}}$, 以及 $\X_{\text{std}}$ 对应的协方差矩阵.\\
    (\textbf{Hint:} 相比中心化, 标准化还需要额外除以标准差.)
    \item[(3)] \textbf{[7pts]} 试求协方差矩阵对应的特征值, 以及投影矩阵 $\W^\star$. 
    \item[(4)] \textbf{[5pts]} 如果选择重构阈值 $t=95\%$, 试求 PCA 后样本 $\X_{\text{std}}$ 在新空间的坐标矩阵.
\end{enumerate}


 

\begin{solution}
此处用于写解答(中英文均可)
~\\
(1) 样本数据各维的均值、标准差:
\begin{align*}
\mu_1 &= \frac{1}{6}\sum_{i=0}^{6}x_{1i} = 4, \\
\mu_2 &= \frac{1}{6}\sum_{i=0}^{6}x_{2i} = 5, \\
\sigma_1 &= \sqrt{\frac{1}{6}\sum_{i=0}^{6}(x_{1i} - \mu_1)^2} = \sqrt{\frac{8}{3}} \approx 1.633, \\
\sigma_2 &= \sqrt{\frac{1}{6}\sum_{i=0}^{6}(x_{2i} - \mu_2)^2} = \sqrt{\frac{10}{3}} \approx 1.826.
\end{align*}
(2) 标准化后的样本矩阵 $\X_{\text{std}} \leftarrow \frac{\x_i - \mu}{\sigma} =$
$$
\begin{bmatrix}
-{\frac{\sqrt{6}}{2}} & -{\frac{\sqrt{6}}{4}} & -{\frac{\sqrt{6}}{4}} & 0  & {\frac{\sqrt{6}}{4}} & {\frac{3\sqrt{6}}{4}}\\
-{\frac{3\sqrt{30}}{10}} & -{\frac{\sqrt{30}}{10}} & 0 & 0  & {\frac{\sqrt{30}}{10}} & {\frac{3\sqrt{30}}{10}}\\
\end{bmatrix}
$$

(3) 协方差矩阵
$$
\frac{1}{5}\X\X^T = \frac{1}{5}
\begin{bmatrix}
6 & {\frac{51\sqrt{5}}{20}} \\
{\frac{51\sqrt{5}}{20}} & 6 \\

\end{bmatrix}
=
\begin{bmatrix}
1.2 & {\frac{51\sqrt{5}}{100}} \\
{\frac{51\sqrt{5}}{100}} & 1.2 \\
\end{bmatrix}
$$
故,计算特征值:$\X\X^T \w = \lambda \w$,解得:$\lambda_1 = \frac{51\sqrt{5}}{100} \approx 2.340$, $\lambda_2 = 1.2 - \frac{51\sqrt{5}}{100} \approx 0.057$
$$\W^* = 
\begin{bmatrix}
1 & -1 \\
1 & 1 \\
\end{bmatrix}
$$
(4) 由于重构阈值为 $0.95$,而 $\frac{\lambda_1}{\lambda_1 + \lambda_2} \approx 0.98 > 0.95$,所以选取 $\lambda_1$ 对应的特征向量作为投影向量,所以有:$\X'_{\text{std}} = \w_{\text{1}}^T\X_{\text{std}} =$ 
$$
\begin{bmatrix}
-{\frac{\sqrt{6}}{2}}-{\frac{3\sqrt{30}}{10}} & -{\frac{\sqrt{6}}{4}}-{\frac{\sqrt{30}}{10}} & -{\frac{\sqrt{6}}{4}}& 0 & {\frac{\sqrt{6}}{4}}+{\frac{\sqrt{30}}{10}} & {\frac{3\sqrt{6}}{4}}+{\frac{3\sqrt{30}}{10}}  
\end{bmatrix}
$$
\end{solution}


\newpage
\section{[25pts] Support Vector Machines}
核函数是 SVM 中常用的工具,其在机器学习中有着广泛的应用与研究. 请仔细阅读学习《机器学习》第六章, 并回答如下问题.
\begin{enumerate}
	\item[(1)] \textbf{[6pts]} 试判断下图 $\textcircled{1}$ 到 $\textcircled{6}$ 中哪些为支持向量.
        \begin{figure}[!htbp]
		\centering
			\includegraphics[width=0.48\textwidth]{svm.pdf}\\
			\caption{分离超平面示意图}
	\end{figure}
        \item[(2)] \textbf{[5pts]} 试判断 $\kappa(\x, \z) = \left(\langle\x, \z\rangle + 1\right)^2$ 是否为核函数,并给出证明或反例.
        \item[(3)] \textbf{[5pts]} 试判断 $\kappa(\x, \z) = \left(\langle\x, \z\rangle - 1\right)^2$ 是否为核函数,并给出证明或反例.
	\item[(4)] \textbf{[9pts]} 试证明:若 $\kappa_1$ 和 $\kappa_2$ 为核函数, 则两者的直积
	\[
	\kappa_1 \otimes \kappa_2(\x, \z)=\kappa_1(\x, \z) \kappa_2(\x, \z)
	\]
	也是核函数. 即证明 《机器学习》(6.26) 成立.

 (\textbf{Hint:} 利用核函数与核矩阵的等价性.)

	
\end{enumerate}

\begin{solution}
此处用于写解答(中英文均可)
~\\
(1)支持向量为:$\textcircled{1}$\\
(2) 
证明 $\kappa(\x, \z) = (\langle \x, \z \rangle + 1)^2$ 是核函数:

由于 $\langle x_i, x_j \rangle$ 为线性核函数, 所以其核矩阵:
$$\K_0 =
\begin{bmatrix}

\langle \x_1, \x_1 \rangle  & \langle \x_1, \x_2 \rangle & \cdots & \langle \x_1, \x_m \rangle \\
\langle \x_2, \x_1 \rangle & \langle \x_2, \x_2 \rangle & \cdots & \langle \x_2, \x_m \rangle \\
\vdots & \vdots & \ddots & \vdots \\
\langle \x_m, \x_1 \rangle & \langle \x_m, \x_2 \rangle & \cdots & \langle \x_m, \x_m \rangle \\
\end{bmatrix}
\succeq 0 $$


即:对于所有的 $y \in R^m$, 以及所有的 $ D = \{\x_1, \x_2, \ldots, \x_m\}$ 有 $y^\top \K_0 y \geq 0$. 而对于 $\kappa(\x, \z) =(\langle \x, \z \rangle + 1)^2$, 其核矩阵为:
$$\K =
\begin{bmatrix}
\langle \x_1, \x_1 \rangle^2 + 2\langle \x_1, \x_1 \rangle + 1 & \langle \x_1, \x_2 \rangle^2 + 2\langle \x_1, \x_2 \rangle + 1 & \cdots & \langle \x_1, \x_m \rangle^2 + 2\langle \x_1, \x_m \rangle + 1 \\
\langle \x_2, \x_1 \rangle^2 + 2\langle \x_2, \x_1 \rangle + 1 & \langle x_2, x_2 \rangle^2 + 2\langle \x_2, \x_2 \rangle + 1 & \cdots & \langle \x_2, \x_m \rangle^2 + 2\langle \x_2, \x_m \rangle + 1 \\
\vdots & \vdots & \ddots & \vdots \\
\langle \x_m, \x_1 \rangle^2 + 2\langle \x_m, \x_1 \rangle + 1 & \langle \x_m, \x_2 \rangle^2 + 2\langle \x_m, \x_2 \rangle + 1 & \cdots & \langle \x_m, \x_m \rangle^2 + 2\langle \x_m, \x_m \rangle + 1
\end{bmatrix}$$
$$= 
\begin{bmatrix}

\langle \x_1, \x_1 \rangle^2  & \langle \x_1, \x_2 \rangle^2 & \cdots & \langle \x_1, \x_m \rangle^2 \\
\langle \x_2, \x_1 \rangle^2 & \langle \x_2, \x_2 \rangle^2 & \cdots & \langle \x_2, \x_m \rangle^2 \\
\vdots & \vdots & \ddots & \vdots \\
\langle \x_m, \x_1 \rangle^2 & \langle \x_m, \x_2 \rangle^2 & \cdots & \langle \x_m, \x_m \rangle^2 \\
\end{bmatrix}
+\K_0 + \textbf{1}
$$
将上式中的第一个矩阵设为 $\K_1$, 显然其等价为核函数 $\langle \x_i, \x_j \rangle^2$ 的核矩阵, 其为多项式核, 故显然 $\K_1$ 为半正定矩阵, 而对于全 1 方阵 \textbf{1}, 也有 $y^\top \textbf{1} y = \sum_{i}\sum_{j} y_i y_j = (\sum_{i} y_i)^2 \geq 0$, 显然 \textbf{1} 也为半正定矩阵, 所以半正定矩阵之和矩阵 $\K$ 也为半正定矩阵, 所以 $\kappa(\x, \z) = (\langle \x, \z \rangle + 1)^2$ 为核函数

(3) 举反例如下:
令 $D = \{[1, 0], [0, 1]\}$, 对于 $\kappa(\x, \z) = (\langle \x, \z \rangle - 1)^2$, 其核矩阵为
$$\K =
\begin{bmatrix}
0 & 1 \\
1 & 0
\end{bmatrix}
$$
$|K| = -1$, 显然 $K$ 不是半正定矩阵, 故 $\kappa(\x, \z) = (\langle \x, \z \rangle - 1)^2$ 不是核函数\\

(4)若 $\kappa_1$ 和 $\kappa_2$ 为核函数, 假设两者的核矩阵分别为 $\K_1$ 和 $\K_2$ 则两者的直积 $\kappa_1 \otimes \kappa_2(\x, \z) = \kappa_1(\x, \z) \kappa_2(\x, \z)$, 其有核矩阵 $\K = \K_1 \odot \K_2$, 其中 $\odot$ 表示两矩阵对应元素相乘运算. 由于核矩阵都是实对称矩阵, 所以将 $\K_2$ 正交分解化可得:
$$\K_2 = \P \A \P^\top$$
其中 $\A$ 为对角线是 $\K_2$ 特征值构成的对角阵, 而 $\P$ 为所对应的特征向量 $\x_i$ 组成的正交矩阵, 所以有:
$$\K_2 = \P \A \P^T = [\x_1, \x_2, \ldots, \x_m] \A [\x_1, \x_2, \ldots, \x_m]^T = \sum_{i=0}^{m} \lambda_i \x_i \x_i^T$$
而显然:
$$\K_1 \odot (\x_i \x_i^T) = \D_i \K_1 \D_i^T$$
其中 $\D_i$ 是对角线为 $\x_i$ 的对角阵, 上式基于对角阵左乘和右乘分别对应, 按行和列乘上相应对角线元素.所以有
$$\K = \K_1 \odot \K_2 = \K_1 \odot \sum_{i=0}^{m} \lambda_i \x_i \x_i^T = \sum_{i=0}^{m} \lambda_i \D_i \K_1 \D_i^\top$$
由于半正定矩阵特征值 $\lambda_i \geq 0$, 所以, 对于所有的 $y \in R^m$, 有:
$$y \K y^T = \sum_{i=0}^{m} \lambda_i \D_i y \K_1 (\D_i y)^T \geq 0$$
所以 $\K$ 为半正定矩阵, 所以 $\kappa_1 \otimes \kappa_2(\x, \z) = \kappa_1(\x, \z) \kappa_2(\x, \z)$ 为核函数。
\end{solution}

\newpage

\section{[30pts] Basics of Neural Networks}

多层前馈神经网络可以被用作分类模型. 在本题中, 我们先回顾前馈神经网络的一些基本概念, 再利用 Python 实现一个简单的前馈神经网络以进行分类任务.

\textbf{[基础原理]} 首先, 考虑一个多层前馈神经网络, 规定网络的输入层是第 $0$ 层, 输入为 $\mathbf{x} \in \mathbb{R}^d$. 网络有 $M$ 个隐层, 第 $h$ 个隐层的神经元个数为 $N_h$, 输入为 $\mathbf{z}_h\in \mathbb{R}^{N_{h-1}}$, 输出为 $\mathbf{a}_h \in \mathbb{R}^{N_h}$, 权重矩阵为 $\mathbf{W}_h \in \mathbb{R}^{N_{h-1} \times N_{h}}$, 偏置参数为 $\mathbf{b}_h \in \mathbb{R}^{N_h}$. 网络的输出层是第 $M+1$ 层, 神经元个数为 $C$, 权重矩阵为 $W_{M+1} \in \mathbb{R}^{N_M \times C}$, 偏置参数为 $\mathbf{b}_{M+1} \in \mathbb{R}^C$, 输出为 $\mathbf{y} \in \mathbb{R}^C$. 网络隐层和输出层的激活函数均为 $f$, 网络训练时的损失函数为 $\mathcal{L}$, 且 $f$ 与 $\mathcal{L}$ 均可微.

\begin{enumerate}
    \item [(1)] \textbf{[5pts]} 请根据前向传播原理, 给出 $\mathbf{z}_h, \mathbf{a}_h\ (1\leq h \leq M)$ 及 $\mathbf{y}$ 的具体数学表示.
    \item [(2)] \textbf{[5pts]} 结合 (1) 的表示形式, 谈谈为何要在神经网络中引入 (非线性) 激活函数 $f$?
\end{enumerate}

\textbf{[编程实践]} 下面, 我们针对一个特征数 $d=2$, 类别数为 $2$ 的分类数据集, 实现一个结构为 ``2-2-1'' 的简单神经网络, 即: 输入层有 $2$ 个神经元; 隐层仅一层, 包含 $2$ 个神经元; 输出层有 $1$ 个神经元; 所有层均使用 Sigmoid 作为激活函数. 此外, 我们使用 BP 算法进行神经网络的训练. \textbf{关于本题的细节介绍及具体要求, 请见附件: \href{https://www.lamda.nju.edu.cn/ML2024Spring/homework/HW3/p3_guide.pdf}{\texttt{p3\hspace{0em}\_\hspace{0em}编程题说明}}.} 请参考编程题说明文档与附件中的代码模板, 完成下面的任务.

\begin{enumerate}
    \item [(3)] \textbf{[15pts]} 基于 \texttt{p3\_models.py}, 补全缺失代码, 实现神经网络分类器的训练与预测功能.
    \item [(4)] \textbf{[5pts]} 参考《机器学习》及第一次作业中对超参数调节流程的介绍, 为 (1) 中模型设置合适的超参数 (即: 学习率与迭代轮数). \textbf{请将选择的超参数设置为调用模型时的默认参数, 并在解答区域简要介绍你的超参数调节流程.}\\(提示: 可以从数据集划分方法, 评估方法, 候选超参数生成方法等角度说明).
\end{enumerate}

\begin{solution} 此处用于写解答(中英文均可)
    ~\\
(1)在神经网络中,每一层的输出都是下一层的输入。对于第 $h$ 层(隐藏层)来说,它的输入 $\mathbf{z}_h$ 是上一层的输出 $\mathbf{a}_{h-1}$ 与权重矩阵 $\mathbf{W}_h$ 的乘积加上偏置向量 $\mathbf{b}_h$,即:

$$\mathbf{z}_h = \mathbf{W}_h \mathbf{a}_{h-1} + \mathbf{b}_h$$

然后通过激活函数 $f$,可以得到该层的输出 $\mathbf{a}_h$:

$$\mathbf{a}_h = f(\mathbf{z}_h)$$

输出层的计算方式与隐藏层类似,只不过输入是最后一个隐藏层的输出,即:

$$\mathbf{y} = f(\mathbf{W}_{M+1} \mathbf{a}_M + \mathbf{b}_{M+1})$$\\
(2)引入非线性激活函数 $f$ 的主要原因是线性模型的表达能力有限。如果神经网络只有线性操作(即没有激活函数或者激活函数是线性的),无论网络有多少层,其总体仍然是一个线性模型,这意味着它只能表示一些简单的模式,不能捕捉数据的复杂特征。而非线性激活函数可以帮助神经网络学习到数据中的非线性模式,从而大大提高了神经网络的表达能力和拟合能力。\\
(3)
 \begin{figure}[H]
    \centering
    \includegraphics[width=0.8\textwidth]{屏幕截图 2024-05-06 150830.png}\\
    \caption{训练与预测结果}
    \label{fig:roc}
\end{figure}
(4)超参数调节流程:\\
\textbf{1. 数据集划分}:\\
使用train\_test\_split方法从sklearn.model\_selection进行数据划分,比如将数据集划分为80$\%$的训练集和20$\%$的验证集。\\
\textbf{2. 评估方法}\\
将 NeuralNetworkClassifier 包装成一个符合 GridSearchCV 要求的评估器。使用GridSearchCV来进行超参数的网格搜索,尝试所有可能的超参数组合,并通过交叉验证来评估每一组超参数的性能。\\
\textbf{3. 超参数候选生成}\\
对于learning\_rate和max\_epoch,设置如下等候选值:\\
learning\_rate: 0.001, 0.01, 0.1\\
max\_epoch: 10, 50, 100, 75, 85, 55, 65, 95\\
\textbf{4. 交叉验证与选择并对model进行更新}\\
使用 GridSearchCV 进行交叉验证,确保模型的泛化能力,根据结果选择最优参数。\\
\textbf{5. 对model进行更新}\\
最优参数为$learning\_rate=0.01 max\_epoch=85$\\
更新模型默认参数,以便在实际使用时能够直接应用最优参数。
\end{solution}

\newpage

\section{[20(+5)pts] Neural Networks with PyTorch}

\begin{tcolorbox}
在上一题的编程实践中, 我们使用 Python 实现了一个简单的神经网络分类器. 其中, 我们根据 BP 算法中神经网络参数梯度的数学定义, 手动实现了梯度计算及参数更新的流程. 然而, 在现实任务中, 我们往往利用\textit{深度学习框架}来进行神经网络的开发及训练. 一些常用的框架例如: PyTorch, Tensorflow 或 JAX, 以及国产的 PaddlePaddle, MindSpore. 这类框架往往支持\textit{自动微分}功能, 仅需定义神经网络的具体结果与前向传播过程, 即可在训练时自动计算参数的梯度, 进行参数更新. 此外, 我们可以使用由框架实现的更成熟的优化器 (如 Adam 等) 来提高模型的收敛速度, 或使用 GPU 加速以提高训练效率. 如果希望在今后的学习科研中应用神经网络, 了解至少一种框架的使用方式是极为有益的.
\end{tcolorbox}

在本题中, 我们尝试使用 PyTorch 框架来进行神经网络的开发, 完成 FashionMNIST 数据集上的图像分类任务. 与上一题考察神经网络底层原理不同, 本题考察大家阅读文档, 搭建模型并解决实际任务的能力. \textbf{关于本题的细节介绍及具体要求, 请见附件: \href{https://www.lamda.nju.edu.cn/ML2024Spring/homework/HW3/p4_guide.pdf}{\texttt{p4\hspace{0em}\_\hspace{0em}编程题说明}}.} 请参考编程题说明文档与附件中的代码模板, 完成下面的任务.

\begin{enumerate}
    \item [(1)] \textbf{[10pts]} 阅读文档, 配置 PyTorch 环境, 补全 \texttt{p4\_models.py} 中神经网络的 \texttt{\_\_init\_\_} 与 \texttt{forward} 方法, 最终成功运行 \texttt{p4\_main.py}. \textbf{请在解答区域附上运行 \texttt{p4\_main.py} 后生成的 \texttt{plot.png}.}
    \item [(2)] \textbf{[10pts]} 从 (1) 中生成的训练过程图片 \texttt{plot.png} 中可以看出: 模型明显出现了\textbf{过拟合}现象, 即训练一定轮次后, 训练集 loss 持续下降, 但测试集 loss 保持不变或转为上升. \\请提出\textbf{至少两种}缓解过拟合的方法, 分别通过编程实现后, 在解答区域附上应用前后的训练过程图片, 并结合图片简要分析方法有效/无效的原因.\\(提示: 可以考虑的方法包括但不限于: Dropout, 模型正则化, 数据增强等.)
    \item [(3)] \textbf{[5pts]} (本题为附加题, 得分计入卷面分数, 但本次作业总得分不超过 100 分)\\
    寻找最优的改进神经网络结构及训练方式的方法, 使模型在另一个未公开的测试集上取得尽可能高的分类准确率.\\本题得分规则如下: \textit{假设共有 $N$ 名同学完成本题, 我们将这 $N$ 名同学的模型测试集分类准确率由高到低排列, 对前 $K=\min\left(\lfloor N/10\rfloor, 10\right)$ 名同学奖励附加题分数. 对于排列序号为 $i$ 的同学($1 \leq i \leq K$), 得分为: $5 - \lfloor 5(i-1)/k \rfloor$.}\\
    (提示: 你可以自由尝试修改模型结构, 修改优化器超参数等方法.)
\end{enumerate}

\begin{solution} 此处用于写解答(中英文均可)
    ~\\
(1)
 \begin{figure}[H]
    \centering
    \includegraphics[width=0.8\textwidth]{plot.png}\\
    \caption{plot}
    \label{fig:roc}
\end{figure}
(2)方法一:添加 Dropout 层,实现为BetterFashionClassifier1类\\
   在神经网络的每个隐藏层之后添加 Dropout 层,以一定概率随机丢弃部分神经元的输出,减少过拟合。
    \begin{figure}[H]
    \centering
    \includegraphics[width=0.8\textwidth]{Dropout 层plot.png}\\
    \caption{Dropout 层plot}
    \label{fig:roc}
\end{figure}
   从图片可以看出,test loss显著降低,与train loss 的差距明显变小,所以方法有效\\
   原因:Dropout 可以随机丢弃神经元的输出,迫使网络学习更加鲁棒的特征表示,减少过拟合。\\
方法二: 在损失函数中添加 L2 正则化项,惩罚模型参数的大小,防止过拟合。实现为BetterFashionClassifier2类
    \begin{figure}[H]
    \centering
    \includegraphics[width=0.8\textwidth]{L2 正则化plot.png}\\
    \caption{L2 正则化plot}
    \label{fig:roc}
\end{figure}
从图片可以看出,test loss与train loss没有发生显著变化,所以方法无效\\
原因:正则化系数可能过大,可能会导致模型欠拟合,影响了模型的拟合能力。\\
(3)解答如下:\\
BetterFashionClassifier类使用了卷积层、池化层、全连接层等组件,用于实现更复杂的神经网络结构。具体来说:\\
- 神经网络包括两个卷积层(Conv2d)、两个最大池化层(MaxPool2d)、两个全连接层(Linear)以及激活函数(ReLU)。\\
- 在 forward 方法中,定义了神经网络的前向传播过程,按照卷积、激活、池化、全连接的顺序进行计算。\\
BetterTrainer 类包括了训练步骤 train\_step 和测试方法 test。\\
- 训练过程中使用交叉熵损失函数(CrossEntropyLoss)和 Adam 优化器(Adam)进行模型优化。\\
- 在测试过程中,计算模型在测试集上的分类准确率。\\
这样设计的神经网络结构和训练方法可以帮助提高模型的分类效果,通过卷积和池化层提取更丰富的特征信息,同时使用交叉熵损失函数和 Adam 优化器进行训练,有助于提高模型的收敛速度和准确率。
\end{solution}

\newpage

\section*{Acknowledgments}
允许与其他同样未完成作业的同学讨论作业的内容, 但需在此注明并加以致谢; 如在作业过程中, 参考了互联网上的资料, 且对完成作业有帮助的, 亦需注明并致谢.

\end{document}